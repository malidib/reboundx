%===============================================================================
%  LaTeX  DOCUMENTATION -- v2.1 (extended, ~6 pages @ 11 pt)
%  “Binary‑Star Modules”  (REBOUNDx ≥ 3.12, rev. B   10 Jul 2025)
%===============================================================================
\documentclass[11pt]{article}
\usepackage[a4paper,margin=2.3cm]{geometry}
\usepackage{amsmath,amssymb,amsfonts,bm}
\usepackage{graphicx}
\usepackage{booktabs}
\usepackage{enumitem}
\usepackage{listings}
\usepackage{hyperref}
\usepackage{natbib}
\hypersetup{colorlinks=true,linkcolor=blue,citecolor=blue,urlcolor=blue}
\DeclareUnicodeCharacter{00A0}{\space} % NBSP
\DeclareUnicodeCharacter{202F}{\space} % NARROW NO-BREAK SPACE
\DeclareUnicodeCharacter{2009}{\space} % THIN SPACE
\DeclareUnicodeCharacter{200A}{\space} % HAIR SPACE
\DeclareUnicodeCharacter{205F}{\space} % MEDIUM MATH SPACE
\DeclareUnicodeCharacter{2007}{\space} % FIGURE SPACE
\DeclareUnicodeCharacter{FEFF}{\space} % ZERO WIDTH NBSP (BOM)
\DeclareUnicodeCharacter{200B}{}       % ZERO WIDTH SPACE -> remove
\DeclareUnicodeCharacter{2060}{}       % WORD JOINER -> remove

\lstset{basicstyle=\ttfamily\footnotesize,
        keywordstyle=\color{blue},
        commentstyle=\color{gray},
        columns=fullflexible,
        keepspaces=true,
        frame=single}

\begin{document}

\title{\textsc{Binary‑Star Evolution Modules in
         \textnormal{REBOUNDx}}}
\author{Mohamad Ali-Dib}
\date{}
\maketitle
\vspace*{-1.5em}

%-------------------------------------------------------------------------------
\begin{abstract}
We document the binary‑star evolution modules distributed with
\textsc{REBOUNDx} $\ge$ 3.12 (revision B, 10 Jul 2025).
The Roche‑lobe mass transfer operator combines overflow, non‑conservative
systemic winds, and common‑envelope drag within a momentum‑conserving
framework featuring adaptive sub‑stepping and robust merge guards.
Separate operators implement magnetic braking, isotropic stellar‑wind
mass loss, thermally driven winds, simplified stellar‑evolution scalings,
and post‑Newtonian relativistic corrections up to 2.5PN order with spin--spin couplings; spin--orbit terms can be added via the \texttt{lense\_thirring} effect.  Equations are derived,
parameters are tabulated, and algorithmic
choices are clarified so that this document can serve as an archival,
citable reference.
\end{abstract}

%-------------------------------------------------------------------------------


\section{Introduction}\label{sec:intro}
Close binary stars evolve under a tightly coupled set of physical processes that redistribute mass and angular momentum and, in some regimes, remove orbital energy. Mass exchange through Roche–lobe overflow (RLOF) is triggered once a donor fills its volume–equivalent Roche surface, typically parameterized by the Eggleton formula \citep{Eggleton1983}, while the instantaneous mass–loss response near the inner Lagrange point can be captured by exponential prescriptions calibrated to the photospheric scale height \citep{Ritter1988}. In more extreme phases, unstable mass transfer and orbital shrinkage may lead to common–envelope (CE) evolution, during which hydrodynamic drag within an extended envelope extracts orbital energy and angular momentum \citep{Ostriker1999}. Independent channels of mass and angular–momentum loss operate throughout a binary’s life: cool convective stars experience magnetic braking via magnetized winds \citep{Verbunt1981,Kawaler1988}, giant–branch stars lose mass through stellar winds that correlate with global parameters \citep{Reimers1975}, and compact binaries emit gravitational radiation that drives secular inspiral on relativistic timescales \citep{Peters1964}. Changes in stellar structure modulate these pathways by altering radii, luminosities, and internal moments of inertia \citep{Hurley2000}. Taken together, these effects govern whether a close binary undergoes stable mass exchange, shrinks into contact and merges, or survives to produce compact remnants.

Capturing this multi-physics landscape with a single method remains challenging. Fully three dimensional hydrodynamic calculations can resolve RLOF streams and CE flows, but they are too expensive for long-term evolution or population-scale parameter studies. Conversely, rapid binary–evolution formalisms encode RLOF, winds, magnetic braking, and gravitational–wave losses through secular prescriptions that are efficient and predictive in isolation, yet they generally assume a two-body context and cannot natively account for higher–order gravitational dynamics (e.g., triples, resonant encounters, or cluster potentials) that can modulate or even trigger mass transfer. A complementary approach is to embed vetted secular and dissipative processes within a high-accuracy $N$-body integrator so that orbital dynamics, spin evolution, and mass exchange proceed self-consistently on a common timestep while preserving the bookkeeping demanded by conservation laws.

This paper presents a suite of new binary–evolution modules implemented in the \textsc{REBOUNDx} extension to the \textsc{REBOUND} $N$-body framework. The modules target the dominant channels relevant to close binaries: (i) a momentum–conserving RLOF operator that combines conservative transfer with non-conservative systemic mass loss and supports multiple specific–angular–momentum prescriptions for escaping gas, with optional donor mass–radius responses following power–law scalings \citep{Eggleton1983,Ritter1988,Hurley2000}; (ii) a magnetic–braking operator implementing the Verbunt–Zwaan/Kawaler torque with a saturation branch appropriate for rapidly rotating convective envelopes \citep{Verbunt1981,Kawaler1988}; (iii) isotropic stellar–wind mass loss following the Reimers scaling \citep{Reimers1975}; (iv) thermally driven (Parker-like) winds parameterized by heating efficiency; and (v) post-Newtonian (PN) corrections providing conservative 2\,PN point–mass and spin–spin terms and 2.5\,PN gravitational–wave radiation reaction \citep{Einstein1915,Peters1964,Kidder1995}; 1\,PN and spin--orbit (1.5\,PN) effects are available via the \texttt{gr\_full} and \texttt{lense\_thirring} modules. Each operator is designed to be unit–agnostic and exposes minimal, physically interpretable parameters that can be calibrated or varied systematically.

Algorithmically, the RLOF/CE module tracks all linear–momentum channels associated with accretion and systemic mass loss and applies a minimal corrective kick to maintain angular–momentum consistency, while adaptive sub-stepping limits the fractional changes in mass and separation per internal step to ensure numerical stability near contact. CE drag is modeled with a Mach-number–dependent dynamical-friction prescription capped by a CFL-like limiter to avoid excessive velocity impulses \citep{Ostriker1999}. The wind and magnetic-braking operators act directly on particle masses and spin vectors, respectively, with safeguards against overshoot in the spin update. The PN module follows the harmonic-gauge two-body equations of motion \citep{Kidder1995} and is split into configurable orders, enabling controlled experiments that isolate conservative precession from dissipative inspiral.

Our goal is to make these processes available, documented, and reproducible within a single $N$-body environment so that studies of close-binary evolution can move seamlessly between isolated binaries and dynamically rich contexts. The remainder of this paper details the physical models and parameterizations (Section~\ref{sec:PhysicalModel}), the numerical implementation and safeguards (Section~\ref{sec:AlgorithmicFlow}), and example applications and recommended usage domains (Section~\ref{sec:future}). By anchoring each operator to established literature \citep{Eggleton1983,Ritter1988,Reimers1975,Verbunt1981,Kawaler1988,Peters1964,Kidder1995,Hurley2000,Ostriker1999}, we aim to provide a citable reference implementation suitable for precision studies of close binaries.



%-------------------------------------------------------------------------------
\section{Effects implementation}


\subsection{Simplified Stellar Evolution}
\label{sec:sse}
Some of the effects implemented below depends on having self-consistent evolution of the stellar radius and luminosity as a function of mass.

To supply mass-dependent stellar properties the optional
\texttt{stellar\_evolution\_sse} operator updates each star's radius and
luminosity using flexible power-law scalings inspired by \citet{Hurley2000}. For a
particle of mass $M$ (in $M_\odot$) we set
\[
R = R_{\rm coeff} R_\odot \left(\frac{M}{M_\odot}\right)^{R_{\rm exp}},\qquad
L = L_{\rm coeff} L_\odot \left(\frac{M}{M_\odot}\right)^{L_{\rm exp}}.
\]

The mass ratio $M/M_\odot$ uses the operator-level parameter
\texttt{sse\_Msun} (default 1) to convert code masses into solar masses;
\texttt{sse\_Rsun} and \texttt{sse\_Lsun} analogously define the solar radius
and luminosity in code units.  Each particle may override the scaling using the
particle-level parameters \texttt{sse\_R\_coeff}, \texttt{sse\_R\_exp},
\texttt{sse\_L\_coeff}, and \texttt{sse\_L\_exp}; if unspecified, the default
values in Table~\ref{tab:sse} are applied.  The particle's radius \texttt{sim.particles[index].r} field is set by $R$, and the luminosity $L$ is written to the attribute
\texttt{sse\_L}. This stored luminosity can be used by other modules such as
the thermally driven wind operator. Virtual particles are skipped, and stellar
masses remain unchanged.

\begin{table}[h]
\centering\footnotesize
\caption{Simplified stellar-evolution parameters}
\label{tab:sse}
\begin{tabular}{@{}llll@{}}
\toprule
Name (scope) & Unit & Default & Purpose \\
\midrule
\texttt{sse\_Msun}   (op) & mass      & 1   & Solar mass in code units\\
\texttt{sse\_Rsun}   (op) & length     & 1   & Solar radius in code units\\
\texttt{sse\_Lsun}   (op) & luminosity & 1   & Solar luminosity in code units\\
\texttt{sse\_R\_coeff} (part) & —       & 1   & Radius scaling prefactor\\
\texttt{sse\_R\_exp}   (part) & —       & 0.8 & Radius mass exponent\\
\texttt{sse\_L\_coeff} (part) & —       & 1   & Luminosity scaling prefactor\\
\texttt{sse\_L\_exp}   (part) & —       & 3.5 & Luminosity mass exponent\\
\bottomrule
\end{tabular}
\end{table}

\subsection{Reimers Stellar Wind Mass Loss}
\label{sec:swml}
In the Reimers prescription \citep{Reimers1975}, isotropic winds remove mass from single cool, low- to intermediate-mass giants. For a star of mass $M$, luminosity $L$, and radius
$R$, the mass-loss rate is
\[
\dot M = -4\times10^{-13}\,\eta\,\frac{L}{L_\odot}\frac{R}{R_\odot}\frac{M_\odot}{M}
\;M_\odot\,\mathrm{yr}^{-1},
\]
where the dimensionless efficiency $\eta$ and luminosity $L$ (\texttt{sse$\_$L} defined above) are specified per particle, while $R$ is taken from the particle's radius field.
Mass is removed isotropically with no linear-momentum
recoil; virtual particles are ignored and after mass loss the system is
recentred on the centre of mass.  A safety limiter caps the fractional
mass change to \texttt{swml\_max\_dlnM} (default $0.1$) per call.  The
prefactor, solar reference values, and the year length can be adjusted via
operator parameters (Table~\ref{tab:swml}).  By default the operator
suppresses winds when a particle is flagged as inside a common envelope or
undergoing Roche–lobe overflow; the switches
\texttt{swml\_disable\_in\_CE} and \texttt{swml\_disable\_in\_RLOF}
control this behaviour by reading the per-particle flags
\texttt{inside\_CE} and \texttt{rlof\_active}.

\begin{table}[h]
\centering\footnotesize
\caption{Stellar wind mass-loss parameters}
\label{tab:swml}
\begin{tabular}{@{}llll@{}}
\toprule
Name (scope) & Unit & Default & Purpose \\
\midrule
\texttt{swml\_eta} (part) & — & — & Wind efficiency $\eta$\\
\texttt{swml\_const} (op) & $M_\odot$/yr & $4\times10^{-13}$ & Reimers prefactor\\
\texttt{swml\_Msun}  (op) & mass & 1 & Solar mass in code units\\
\texttt{swml\_Rsun}  (op) & length & 1 & Solar radius in code units\\
\texttt{swml\_Lsun}  (op) & luminosity & 1 & Solar luminosity in code units\\
\texttt{swml\_year}  (op) & time & 1 & Length of Julian year in code units\\
\texttt{swml\_max\_dlnM} (op) & — & 0.1 & Max $|\Delta M|/M$ per call\\
\texttt{swml\_disable\_in\_CE} (op) & bool & 1 & Skip if \texttt{inside\_CE}$>0$\\
\texttt{swml\_disable\_in\_RLOF} (op) & bool & 1 & Skip if \texttt{rlof\_active}$>0$\\
\bottomrule
\end{tabular}
\end{table}

\subsection{Parker-type thermal wind}
\label{sec:tdw}

In cool, low-gravity stars, thermal pressure can launch isotropic winds that carry away mass \citep{parker}. For a star
of mass $M$, luminosity $L$, and radius $R$, we adopt a Parker-like scaling
\[
\dot M = -C_{\rm th}\,\eta\,\left(\frac{R}{R_\odot}\right)^{\alpha_R}
                   \left(\frac{L}{L_\odot}\right)^{\alpha_L}
                   \left(\frac{M_\odot}{M}\right)^{\alpha_M}
\;M_\odot\,\mathrm{yr}^{-1},
\]
where $\eta$ is a dimensionless heating efficiency and the exponents have
defaults $(\alpha_R,\alpha_L,\alpha_M)=(2,\tfrac{3}{2},1)$.  The luminosity
$L$ is read from the particle attribute \texttt{sse\_L}, which must be
populated either by the simplified stellar‑evolution operator or manually by
the user, while $R$ and $M$ are taken from the particle's $r$ and $m$ fields.
Mass is removed isotropically
with no linear-momentum recoil; virtual particles are ignored, and after mass
loss the system is recentred on the centre of mass.  The prefactor, solar
reference values, exponents, maximum fractional mass change, and the year
length can be adjusted via operator parameters (Table~\ref{tab:tdw}). The
operator is unit-agnostic if these scaling constants are specified
consistently.
By default the operator suspends winds for stars marked as being inside a
common envelope or undergoing active Roche--lobe overflow, as indicated by
the per-particle flags \texttt{inside\_CE} and \texttt{rlof\_active}.  This
behaviour is controlled by \texttt{tdw\_disable\_in\_CE} and
\texttt{tdw\_disable\_in\_RLOF}.

\begin{table}[h]
\centering\footnotesize
\caption{Thermally driven wind parameters}
\label{tab:tdw}
\begin{tabular}{@{}llll@{}}
\toprule
Name (scope) & Unit & Default & Purpose \\
\midrule
\texttt{tdw\_eta} (part) & — & — & Wind efficiency $\eta$\\
\texttt{sse\_L}   (part) & luminosity & — & Stellar luminosity $L$\\[0.2em]
\texttt{tdw\_const} (op) & $M_\odot$/yr & $2\times10^{-14}$ & Thermal-wind prefactor $C_{\rm th}$\\
\texttt{tdw\_Msun}  (op) & mass & 1 & Solar mass in code units\\
\texttt{tdw\_Rsun}  (op) & length & 1 & Solar radius in code units\\
\texttt{tdw\_Lsun}  (op) & luminosity & 1 & Reference luminosity $L_\odot$ in code units\\
\texttt{tdw\_year}  (op) & time & 1 & Length of Julian year in code units\\
\texttt{tdw\_alpha\_R} (op) & — & 2 & Exponent $\alpha_R$ on $R/R_\odot$\\
\texttt{tdw\_alpha\_L} (op) & — & $3/2$ & Exponent $\alpha_L$ on $L/L_\odot$\\
\texttt{tdw\_alpha\_M} (op) & — & 1 & Exponent $\alpha_M$ on $M_\odot/M$\\
\texttt{tdw\_max\_dlnM} (op) & — & 0.1 & Max $|\Delta M|/M$ per call\\
\texttt{tdw\_disable\_in\_CE} (op) & bool & 1 & Skip if \texttt{inside\_CE}$>0$\\
\texttt{tdw\_disable\_in\_RLOF} (op) & bool & 1 & Skip if \texttt{rlof\_active}$>0$\\
\bottomrule
\end{tabular}
\end{table}

\subsection{Eddington-Limited Winds}
\label{sec:edw}

Luminosities exceeding the electron-scattering Eddington limit in some massive and highly luminous stellar objects drive isotropic
mass loss according to
\[
\dot M = -C_{\rm edd}\,\max\!\left(0,\frac{L}{L_{\rm Edd}}-1\right)
\;M_\odot\,\mathrm{yr}^{-1},
\]
where $L_{\rm Edd} = L_{\rm Edd,coeff}\,(M/M_\odot)L_\odot$. The luminosity $L$
is read from the particle attribute \texttt{sse\_L}, which must be supplied
by the user or the simplified stellar-evolution operator. Mass is removed
isotropically with no linear-momentum recoil; virtual particles are ignored,
and after mass loss the system is recentred on the centre of mass. A limiter
caps the fractional mass change at \texttt{edw\_max\_dlnM} per call. Operator
parameters controlling the scaling constants are listed in
Table~\ref{tab:edw}.
By default the operator skips stars flagged as being inside a common envelope
or undergoing Roche--lobe overflow, controlled by
\texttt{edw\_disable\_in\_CE} and \texttt{edw\_disable\_in\_RLOF}, which read
\texttt{inside\_CE} and \texttt{rlof\_active}.

\begin{table}[h]
\centering\footnotesize
\caption{Super-Eddington wind parameters}
\label{tab:edw}
\begin{tabular}{@{}llll@{}}
\toprule
Name (scope) & Unit & Default & Purpose \\
\midrule
\texttt{sse\_L} (part) & luminosity & — & Stellar luminosity $L$\\
\texttt{edw\_const} (op) & $M_\odot$/yr & $1\times10^{-6}$ & Mass-loss prefactor $C_{\rm edd}$\\
\texttt{edw\_Msun}  (op) & mass & 1 & Solar mass in code units\\
\texttt{edw\_Lsun}  (op) & luminosity & 1 & Solar luminosity in code units\\
\texttt{edw\_year}  (op) & time & 1 & Length of Julian year in code units\\
\texttt{edw\_Ledd\_coeff} (op) & $L_\odot/M_\odot$ & $3.2\times10^{4}$ & $L_{\rm Edd}$ per unit mass\\
\texttt{edw\_max\_dlnM} (op) & — & 0.1 & Max $|\Delta M|/M$ per call\\
\texttt{edw\_disable\_in\_CE} (op) & bool & 1 & Skip if \texttt{inside\_CE}$>0$\\
\texttt{edw\_disable\_in\_RLOF} (op) & bool & 1 & Skip if \texttt{rlof\_active}$>0$\\
\bottomrule
\end{tabular}
\end{table}
\subsection{Magnetic Braking}
\label{sec:mb}

Low‑mass stars with convective envelopes lose spin angular momentum through
magnetised winds. In tidally coupled binaries (particularly when a component is close to synchronous rotation), tidal torques replenish the braked spin by drawing from the orbital reservoir, leading to a net loss of orbital angular momentum and a secular decrease of the semi‑major axis and orbital period, whereas in wide or weakly coupled systems the orbital response is negligible on comparable timescales.

We implement the Verbunt--Zwaan / Kawaler torque law
\citep{Verbunt1981,Kawaler1988}, applying a braking torque antiparallel to the
spin vector.

\paragraph{Torque law.}
For a star of mass $M$, radius $R$, and angular velocity
$\omega = \lVert\boldsymbol{\Omega}\rVert$, the spin‑down torque is
\begin{equation}
\label{eq:mb_torque}
\frac{dJ}{dt}
= -K\,\Big(\frac{R}{R_\odot}\Big)^{1/2}\Big(\frac{M}{M_\odot}\Big)^{-1/2}
\begin{cases}
\omega^3, & \omega \le \omega_{\rm sat},\\[3pt]
\omega\,\omega_{\rm sat}^{2}, & \omega > \omega_{\rm sat},
\end{cases}
\end{equation}
where $K$ is a normalisation constant (operator parameter \texttt{mb\_K},
default $2.7\times10^{47}$ in cgs). The torque acts colinearly with
$\boldsymbol{\Omega}$, so the spin direction is unchanged by this operator.

\paragraph{Units and scaling.}
Users specify \texttt{mb\_K} in cgs. Internally we convert to code units using
the operator parameters \texttt{mb\_Msun}, \texttt{mb\_Rsun}, and
\texttt{mb\_year}, which define $M_\odot$, $R_\odot$, and the Julian year in
\emph{code} units. Defining the cgs-per-code base units
$M_{\rm unit}=\mathrm{g}/\mathrm{(code\;mass)}$,
$L_{\rm unit}=\mathrm{cm}/\mathrm{(code\;length)}$,
$T_{\rm unit}=\mathrm{s}/\mathrm{(code\;time)}$, and noting that
$[K]=M\,L^{2}\,T$, we obtain
\begin{equation}
K_{\rm code}=\frac{K_{\rm cgs}}{M_{\rm unit}\,L_{\rm unit}^{2}\,T_{\rm unit}},
\end{equation}
and then evaluate Eq.~\eqref{eq:mb_torque} with the explicit
dimensionless factors $(R/R_\odot)^{1/2}(M/M_\odot)^{-1/2}$.
Users do not need to manually rescale \texttt{mb\_K}.

\paragraph{Saturation.}
If a particle provides a convective turnover time \texttt{mb\_tau\_conv},
the code sets the critical angular velocity via the Rossby number as
\begin{equation}
\omega_{\rm sat} = \frac{2\pi}{\texttt{mb\_Rossby\_sat}\times \texttt{mb\_tau\_conv}},
\end{equation}
with \texttt{mb\_Rossby\_sat} (operator level; default 0.1).
A particle-supplied \texttt{mb\_omega\_sat} overrides this value.
For $\omega>\omega_{\rm sat}$ the torque scales linearly with $\omega$,
ensuring a continuous transition at the threshold.

\paragraph{Time integration (closed form).}
Because the torque is colinear with $\boldsymbol{\Omega}$,
the magnitude $\omega$ obeys a scalar ODE with a closed-form solution. Define
\begin{equation}
C \;=\; \frac{K_{\rm code}}{I}\,
\Big(\frac{R}{R_\odot}\Big)^{1/2}\Big(\frac{M}{M_\odot}\Big)^{-1/2}.
\end{equation}
Then
\begin{align}
\text{unsaturated:}\quad
&\frac{d\omega}{dt} = -C\,\omega^3
\;\Rightarrow\;
\omega(t+\Delta t) = \frac{\omega(t)}{\sqrt{1+2C\,\omega(t)^2\,\Delta t}},
\\[4pt]
\text{saturated:}\quad
&\frac{d\omega}{dt} = -C\,\omega_{\rm sat}^2\,\omega
\;\Rightarrow\;
\omega(t+\Delta t) = \omega(t)\,\exp\!\big[-C\,\omega_{\rm sat}^2\,\Delta t\big].
\end{align}
If a step starts in the saturated regime and crosses to the unsaturated regime,
the update is applied piecewise exactly: exponential decay to
$\omega_{\rm sat}$ followed by the unsaturated formula for the remainder of the
step. The spin vector is finally rescaled by
$\boldsymbol{\Omega}\leftarrow \boldsymbol{\Omega}\times[\omega(t+\Delta t)/\omega(t)]$.
This scheme is positivity‑preserving and avoids step‑size instabilities.

\paragraph{Activation and guards.}
Braking is applied only if a particle sets \texttt{mb\_on}$=1$ and
\texttt{mb\_convective}$=1$. Missing parameters, non‑positive or non‑finite
$I$, $M$, or $R$ bypass the update. A per‑particle radius override
\texttt{mb\_R} (in code length units) is supported; otherwise the particle
radius \texttt{p->r} is used. The operator does nothing for vanishing spin
vectors.

\begin{table}[h]
\centering\footnotesize
\caption{Magnetic braking parameters}
\label{tab:mb}
\begin{tabular}{@{}llll@{}}
\toprule
Name (scope) & Unit & Default & Purpose \\
\midrule
\texttt{mb\_K} (op) & cgs & $2.7\times10^{47}$ & Braking constant $K$\\
\texttt{mb\_Msun} (op) & mass & 1 & Solar mass in code units\\
\texttt{mb\_Rsun} (op) & length & 1 & Solar radius in code units\\
\texttt{mb\_year} (op) & time & 1 & Julian year in code units\\
\texttt{mb\_Rossby\_sat} (op) & — & 0.1 & Critical Rossby number\\
\texttt{mb\_on} (part) & bool & 0 & Enable magnetic braking\\
\texttt{mb\_convective} (part) & bool & 0 & Convective‑envelope flag\\
\texttt{mb\_omega\_sat} (part) & 1/t & $\infty$ & Saturation angular velocity\\
\texttt{mb\_tau\_conv} (part) & time & — & Convective turnover time\\
\texttt{mb\_R} (part) & length & — & Radius override (else particle radius)\\
\texttt{I} (part) & mass\,length$^2$ & — & Moment of inertia\\
\texttt{Omega} (part, vec) & 1/t & — & Spin angular‑velocity vector (\texttt{reb\_vec3d})\\
\bottomrule
\end{tabular}
\end{table}

\paragraph{Sanity checks.}
With \texttt{mb\_Msun} = \texttt{mb\_Rsun} = \texttt{mb\_year} = 1, $M=R=1$,
and $\omega$ in rad/year, the unsaturated torque scales as $\omega^3$ with the
expected normalisation from the literature. The update is continuous at
$\omega=\omega_{\rm sat}$ by construction.


\subsection{Post-Newtonian (PN) corrections}
\label{sec:pn}

Compact binaries experience relativistic corrections that couple the spins and
radiate orbital energy. The \texttt{post\_newtonian} effect implements the
harmonic-coordinate point-mass equations of motion from \citet{Kidder1995} for
each massive pair:
\begin{itemize}[nosep,leftmargin=1.8em]
  \item $2$\,PN conservative point-mass (PM) and spin--spin (SS) corrections;
  \item $2.5$\,PN gravitational-wave radiation reaction (RR).
\end{itemize}
Note that we omitted  $1$\,PN and $1.5$\,PN corrections as they can be added via \texttt{Reboundx}' built-in gr$\_$full and lense$\_$thirring effects.


\paragraph{Variables and masses.}
For two bodies with masses $m_i$, $m_j$, separation vector $\mathbf{r}$,
relative velocity $\mathbf{v}$, define
\[
  m = m_i + m_j,\quad
  \mu = \frac{m_i m_j}{m},\quad
  \eta = \frac{\mu}{m},\quad
  \mathbf{n} = \frac{\mathbf{r}}{r},\quad
  \dot r = \mathbf{v}\cdot\mathbf{n}.
\]
Spins $S_1,S_2$ are \emph{physical} angular momenta (units mass\,length$^2$/time).
If users prefer dimensionless spins $\chi_i$, convert via
$S_i = \chi_i\,G m_i^2 / c$ in the simulation's unit system.

\paragraph{2\,PN conservative (Eqs.\,2.2d,e).}
The point-mass part is written as
\[
\mathbf{a}_{2\mathrm{PN}}^{\rm pm}
= -\frac{G\,m}{c^4 r^2}
   \Bigl[A_{2}\,\mathbf{n} + B_{v,2}\,\mathbf{v}\Bigr],
\]
with $A_2$ and $B_{v,2}$ built from $v^2$, $\dot r^2$, and $Gm/r$;
the implementation includes the \(-\tfrac12\,\eta(13-4\eta)\,(Gm/r)\,v^2\) term
so that $A_2$ matches \citet{Kidder1995}. The spin--spin coupling is
\[
  \mathbf{a}_{2\mathrm{PN}}^{\rm ss}
  = -\frac{3\,G}{\mu\,c^2\,r^4}
    \left[
      \bigl(S_1\!\cdot\!S_2 - 5(\mathbf{n}\!\cdot\!S_1)(\mathbf{n}\!\cdot\!S_2)\bigr)\mathbf{n}
      + (\mathbf{n}\!\cdot\!S_2)\,S_1 + (\mathbf{n}\!\cdot\!S_1)\,S_2
    \right],
\]
which uses physical spins and the reduced mass $\mu$ in the overall prefactor.
Hence
\[
  \mathbf{a}_{2\mathrm{PN}}=\mathbf{a}_{2\mathrm{PN}}^{\rm pm}+\mathbf{a}_{2\mathrm{PN}}^{\rm ss}.
\]

\paragraph{2.5\,PN radiation reaction (Eq.\,2.2f).}
\[
\mathbf{a}_{2.5\mathrm{PN}}
= \frac{8\,G^2 m^2 \eta}{5\,c^5 r^3}
\left[
  \dot r\left(18v^2+\tfrac{2}{3}\tfrac{Gm}{r}-25\dot r^2\right)\mathbf{n}
 -\left(6v^2-2\tfrac{Gm}{r}-15\dot r^2\right)\mathbf{v}
\right].
\]

\paragraph{Units and parameters.}
The effect parameter \texttt{c} is the speed of light in \emph{code units}.
The toggles \texttt{pn\_2PN}, and \texttt{pn\_25PN}
(default 1) enable the corresponding sectors. Particle parameter
\texttt{pn\_spin} is a \texttt{reb\_vec3d} storing the physical spin vector.

\paragraph{Momentum conservation and scope.}
Accelerations are built for the relative coordinate and split between bodies
in proportion to their masses, preserving total linear momentum.
Spin precession equations are not included; if spins are misaligned, the
orbital dynamics include spin--spin forces with fixed spins.

\paragraph{Effect on binary orbits.}
The 2.5\,PN term removes orbital energy and angular momentum, driving a secular
decrease of the semi-major axis and (typically) eccentricity toward merger.
When spins are present, the spin–spin terms cause relativistic
precession and can modulate the instantaneous orbital plane and pericentre
orientation, altering waveform phase evolution even when the mean orbital
elements change slowly.



\subsection{Roche–Lobe Overflow and Common–Envelope Operator}
\label{sec:RLOF}

This effect models mass transfer in compact binaries via Roche–lobe overflow
(RLOF) together with optional common–envelope (CE) drag. The implementation
targets robust long integrations: internal sub–stepping constrains
$|\Delta M|/M$ and $|\Delta r|/r$, merge guards prevent divergences, and
diagnostics expose the accumulated energy change from non–Hamiltonian updates.

\paragraph{Roche geometry and mass–loss law.}
The donor’s volume–equivalent Roche radius is evaluated with the Eggleton
formula \citep{Eggleton1983},
\begin{equation}
R_{\rm L} = a\,\frac{0.49\,q^{2/3}}{0.60\,q^{2/3} + \ln(1+q^{1/3})},
\qquad q \equiv \frac{M_{\rm d}}{M_{\rm a}},
\end{equation}
and the instantaneous mass–loss rate follows the Ritter prescription
\citep{Ritter1988},
\begin{equation}
\dot M_{\rm d} = -\dot M_0\,
\exp\!\left[\frac{R_\ast - R_{\rm L}}{H_P}\right],
\end{equation}
with the exponent clamped to $\le 80$ to avoid numerical overflow. The donor
particle provides $H_P$ and $\dot M_0$ as \texttt{rlmt\_Hp} and
\texttt{rlmt\_mdot0}.

\paragraph{Conservative and systemic mass channels.}
Over a sub–step of size $\Delta t$, the donor loses $m_{\rm loss}=-\dot M_{\rm d}\,\Delta t>0$,
which is split into accreted and wind parts:
\[
m_{\rm acc} = (1-f_{\rm loss})\,m_{\rm loss},\qquad
m_{\rm wind} = f_{\rm loss}\,m_{\rm loss},\quad f_{\rm loss}\in[0,1].
\]
The accretor’s mass increases by $m_{\rm acc}$; the wind mass $m_{\rm wind}$
leaves the system.

\paragraph{Linear momentum (exact).}
Conservative accretion is treated internally: $m_{\rm acc}$ is added to the
accretor with the donor’s instantaneous velocity, while the donor’s mass is
reduced; this leaves the pair’s total momentum unchanged. Systemic wind
removes linear momentum $m_{\rm wind}\,\mathbf v_{\rm loss}$; the code applies a
uniform shift to both stars so that
$\Delta \mathbf P = -\,m_{\rm wind}\,\mathbf v_{\rm loss}$ exactly.

\paragraph{$j$–loss (wind angular momentum).}
The specific angular momentum carried by the wind is prescribed by a mode
parameter:
\begin{align*}
\text{mode 0:}~&\mathbf v_{\rm loss}=\mathbf v_{\rm d}\quad\Rightarrow\quad
\Delta\mathbf L_{\rm wind} = m_{\rm wind}\,(\mathbf r_{\rm d}\!-\mathbf R_{\rm CM}) \times \mathbf v_{\rm d},\\
\text{mode 1:}~&\mathbf v_{\rm loss}=\mathbf v_{\rm a}\quad\Rightarrow\quad
\Delta\mathbf L_{\rm wind} = m_{\rm wind}\,(\mathbf r_{\rm a}\!-\mathbf R_{\rm CM}) \times \mathbf v_{\rm a},\\
\text{mode 2:}~&\mathbf v_{\rm loss}=\mathbf v_{\rm CM}\quad\Rightarrow\quad
\Delta\mathbf L_{\rm wind}=\mathbf 0\quad,\\
\text{mode 3:}~&|\Delta\mathbf L_{\rm wind}|=m_{\rm wind}\,f_j\,j_{\rm orb},\qquad
j_{\rm orb}\equiv\frac{J}{M}=\frac{\mu\,|\mathbf r\times\mathbf v|}{M_{\rm d}+M_{\rm a}},
\qquad \mu=\frac{M_{\rm d}M_{\rm a}}{M_{\rm d}+M_{\rm a}}.
\end{align*}
with $\Delta\mathbf L_{\rm wind}$ aligned with the orbital angular momentum.
After the linear momentum update, the code applies a \emph{pure torque}
(zero net impulse) so that the pair’s orbital angular momentum is reduced by
exactly $\Delta\mathbf L_{\rm wind}$.

\paragraph{Conservative transfer angular momentum.}
In the absence of external torques, conservative mass relocation should not
change the system’s orbital angular momentum. The operator computes the
angular–momentum difference between placing $m_{\rm acc}$ at the donor and at
the accretor (both at the donor velocity) and applies a minimal pure‐torque
velocity correction to enforce $\Delta L_{\rm acc}=0$.

\paragraph{Donor radius evolution (optional).}
If the donor supplies \texttt{rlmt\_R\_slope}$=\alpha_R\ne0$, its radius is
updated after each sub–step via a power law
\[
R_{\rm d}(M) = R_{\rm ref}\left(\frac{M}{M_{\rm ref}}\right)^{\alpha_R},
\]
where \texttt{rlmt\_R\_ref\_mass} and \texttt{rlmt\_R\_ref\_radius} default to
the donor’s current values if not provided.

\paragraph{Inter-module flags.}
After each sub-step the operator writes the boolean attributes
\texttt{inside\_CE} and \texttt{rlof\_active} on both donor and accretor.
These indicate whether the accretor lies within the donor's radius and
whether the RLOF channel is actively removing mass. Other effects, such as
stellar winds, inspect these flags (in conjunction with their own disable
switches) to suspend their action during common-envelope phases or while
RLOF is operating.

\subsubsection{Common–Envelope (CE) Drag}
When the accretor is inside the donor ($r<R_\ast$), the code can apply
dynamical friction following Ostriker’s formula,
\[
\mathbf a_{\rm DF} = -\,\frac{4\pi G^2 M_{\rm a}\rho}{v_{\rm rel}^3}\,
I(\mathcal M)\,\mathbf v_{\rm rel},\qquad\mathcal M=\frac{v_{\rm rel}}{c_s},
\]
with
\[
I(\mathcal M)=
\begin{cases}
\frac{1}{3}\mathcal M^3+\frac{1}{5}\mathcal M^5, & \mathcal M<0.02,\\[0.4em]
\frac{1}{2}\ln\!\frac{1+\mathcal M}{1-\mathcal M}-\mathcal M, & 0.02\le\mathcal M<1,\\[0.4em]
\ln(1/x_{\min})+\tfrac12\ln\!\bigl(1-\mathcal M^{-2}\bigr), \quad \mathcal M\ge1,
\end{cases}
\]
capped by $\ln(1/x_{\min})$ for continuity. A geometric term
$-\pi\rho R_{\rm a}^2 Q_d\,v_{\rm rel}\,\mathbf v_{\rm rel}/M_{\rm a}$ may be
added. The envelope structure comes either from a user–supplied table
$(s,\rho,c_s)$ (\texttt{ce\_profile\_file}) with log–log interpolation or from
a power law $\rho=\rho_0 s^{\alpha_\rho}$, $c_s=c_{s0}s^{\alpha_{c_s}}$.
The velocity kick per sub–step is limited by
$|\Delta\mathbf v|\le\texttt{ce\_kick\_cfl}\,c_s$.
By default the envelope is an \emph{external} sink: no opposite reaction is
applied to the donor; this can be enabled with \texttt{ce\_reaction\_on\_donor}=1.

\subsubsection{Numerics and flow}
Each operator call subdivides the requested interval into at least
\texttt{rlmt\_min\_substeps} sub–steps; the size adapts to satisfy
$|\Delta M|/M\le\texttt{rlmt\_substep\_max\_dm}$ and
$|\Delta r|/r\le\texttt{rlmt\_substep\_max\_dr}$. A merge guard triggers when
$r\le\varepsilon_{\rm merge}$ (user parameter \texttt{merge\_eps} or the
default $0.5\min(R_\ast,R_{\rm a})$ with a small positive floor). The operator
purges zero–mass particles and detaches itself if either component vanishes.
The accumulated energy change $\Delta E$ over the call is exported as
\texttt{rlmt\_last\_dE} for diagnostics; because the scheme includes physical
sinks (wind, drag) this value is not expected to vanish.

\subsubsection{Parameters}
\begin{table}[h]
\centering\footnotesize
\begin{tabular}{@{}llll@{}}
\toprule
Name (scope) & Unit & Default & Purpose \\
\midrule
\texttt{rlmt\_donor} (op) & — & — & Donor index (double, cast to int) \\
\texttt{rlmt\_accretor} (op) & — & — & Accretor index (double, cast to int) \\
\texttt{rlmt\_Hp} (part) & length & — & Pressure scale height (donor) \\
\texttt{rlmt\_mdot0} (part) & M/t & — & Reference overflow rate (donor) \\
\texttt{rlmt\_loss\_fraction} (op) & — & 0 & Wind fraction $f_{\rm loss}$ \\
\texttt{jloss\_mode} (op) & — & 0 & Wind $j$ prescription (0–3) \\
\texttt{jloss\_factor} (op) & — & 1 & Scale $f_j$ for mode 3 \\
\texttt{rlmt\_skip\_in\_CE} (op) & bool & 1 & Skip RLOF if $r<R_\ast$ \\
\texttt{rlmt\_substep\_max\_dm} (op) & — & $10^{-3}$ & Max $|\Delta M|/M$ per sub–step \\
\texttt{rlmt\_substep\_max\_dr} (op) & — & $5\times10^{-3}$ & Max $|\Delta r|/r$ per sub–step \\
\texttt{rlmt\_min\_substeps} (op) & int & 3 & Minimum sub–steps \\
\texttt{ce\_profile\_file} (op) & path & — & Table $(s,\rho,c_s)$ for CE \\
\texttt{ce\_rho0}, \texttt{ce\_cs} (op) & dens, speed & — & Power–law CE normalization \\
\texttt{ce\_alpha\_rho}, \texttt{ce\_alpha\_cs} (op) & — & 0 & Power–law slopes \\
\texttt{ce\_xmin} (op) & — & $10^{-4}$ & Coulomb cutoff \\
\texttt{ce\_Qd} (op) & — & 0 & Geometric drag coefficient \\
\texttt{ce\_kick\_cfl} (op) & — & 1 & Velocity–kick limiter \\
\texttt{ce\_reaction\_on\_donor} (op) & bool & 0 & Apply opposite CE kick to donor \\
\texttt{merge\_eps} (op) & length & $0.5\min(R_\ast,R_{\rm a})$ & Merge radius (with floor) \\
\texttt{rlmt\_R\_slope} (part) & — & 0 & Donor mass–radius exponent $\alpha_R$ \\
\texttt{rlmt\_R\_ref\_mass}, \texttt{rlmt\_R\_ref\_radius} (part) & M, L & donor’s & $R(M)$ references \\
\texttt{rlmt\_last\_dE} (op, out) & energy & — & Energy change (diagnostic) \\
\bottomrule
\end{tabular}
\end{table}

\paragraph{Notes.}
All scalar parameters are read as doubles by REBOUNDx and cast to integers for
indices/toggles. If available in your build, a filename in
\texttt{ce\_profile\_file} can be provided to load a CE table; otherwise the
power–law model is used.



%-------------------------------------------------------------------------------
\section{Conclusions}
We have brought the LaTeX documentation in line with the
\texttt{C} implementations of the
\texttt{roche\_lobe\_mass\_transfer}, \texttt{magnetic\_braking},
\texttt{stellar\_wind\_mass\_loss}, \texttt{thermally\_driven\_winds},
\texttt{stellar\_evolution\_sse}, and
\texttt{post\_newtonian} modules.
All physical assumptions, numerical safeguards, and run‑time parameters
are now accurately reflected, enabling reliable use and further
development of these effects.

%-------------------------------------------------------------------------------
\bibliographystyle{plainnat}
\begin{thebibliography}{}
\bibitem[Eggleton(1983)]{Eggleton1983}
  Eggleton,~P.\ 1983, \emph{ApJ}, 268, 368.
\bibitem[Kawaler(1988)]{Kawaler1988}
  Kawaler,~S.~D.\ 1988, \emph{ApJ}, 333, 236.
\bibitem[Reimers(1975)]{Reimers1975}
  Reimers,~D.\ 1975, \emph{Mem.\ Soc.\ R.\ Sci.\ Li\`ege}, 8, 369.
\bibitem[Ostriker(1999)]{Ostriker1999}
  Ostriker,~E.\ 1999, \emph{ApJ}, 513, 252.
\bibitem[Peters(1964)]{Peters1964}
  Peters,~P.\ 1964, \emph{Phys.\ Rev.}, 136, B1224.
\bibitem[Kidder(1995)]{Kidder1995}
  Kidder,~L.~E.\ 1995, \emph{Phys.\ Rev.\ D}, 52, 821.
\bibitem[Einstein(1915)]{Einstein1915}
  Einstein,~A.\ 1915, \emph{Preuss.\ Akad.\ Wiss.\ Berlin}, 831.
\bibitem[Ritter(1988)]{Ritter1988}
  Ritter,~H.\ 1988, \emph{A\&A}, 202, 93.
\bibitem[Verbunt \& Zwaan(1981)]{Verbunt1981}
  Verbunt,~F. \& Zwaan,~C.\ 1981, \emph{A\&A}, 100, L7.
\bibitem[Hurley et~al.(2000)]{Hurley2000}
  Hurley,~J.~R., Pols,~O.~R., \& Tout,~C.~A.\ 2000, \emph{MNRAS}, 315, 543.
  \bibitem[Parker(1958)]{parker} Parker, E.~N.\ 1958, ApJ, 128, 664. doi:10.1086/146579
\end{thebibliography}

\end{document}
