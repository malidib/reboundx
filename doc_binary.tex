%===============================================================================
%  LaTeX  DOCUMENTATION -- v2.1 (extended, ~6 pages @ 11 pt)
%  “Binary‑Star Modules”  (REBOUNDx ≥ 3.12, rev. B   10 Jul 2025)
%===============================================================================
\documentclass[11pt]{article}
\usepackage[a4paper,margin=2.3cm]{geometry}
\usepackage{amsmath,amssymb,amsfonts,bm}
\usepackage{graphicx}
\usepackage{booktabs}
\usepackage{enumitem}
\usepackage{listings}
\usepackage{hyperref}
\usepackage{natbib}
\hypersetup{colorlinks=true,linkcolor=blue,citecolor=blue,urlcolor=blue}

\lstset{basicstyle=\ttfamily\footnotesize,
        keywordstyle=\color{blue},
        commentstyle=\color{gray},
        columns=fullflexible,
        keepspaces=true,
        frame=single}

\begin{document}

\title{\textsc{Binary‑Star Evolution Modules in
         \textnormal{REBOUNDx}}:\\
       Roche‑Lobe Transfer, Magnetic Braking, Stellar Winds, and Relativity}
\author{Operator files: \texttt{roche\_lobe\_mass\_transfer.c},
        \texttt{magnetic\_braking.c},
        \texttt{stellar\_wind\_mass\_loss.c},
        \texttt{post\_newtonian.c}\\
        Revision B – 10 July 2025}
\date{}
\maketitle
\vspace*{-1.5em}

%-------------------------------------------------------------------------------
\begin{abstract}
We document the binary‑star evolution modules distributed with
\textsc{REBOUNDx} $\ge$ 3.12 (revision B, 10 Jul 2025).
The Roche‑lobe mass transfer operator combines overflow, non‑conservative
systemic winds, and common‑envelope drag within a momentum‑conserving
framework featuring adaptive sub‑stepping and robust merge guards.
Separate operators implement magnetic braking, isotropic stellar‑wind
mass loss, and post‑Newtonian relativistic corrections through 2.5PN
order.  Equations are derived, parameters are tabulated, and algorithmic
choices are clarified so that this document can serve as an archival,
citable reference.
\end{abstract}

%-------------------------------------------------------------------------------
\section{Background and Motivation}
\label{sec:intro}
Roche‑lobe overflow, common‑envelope evolution, stellar winds, magnetic
braking, and relativistic effects govern the orbital fates of close
binaries\citep{Eggleton1983,Ostriker1999,Peters1964}.  Hydrodynamic
simulations capture these effects but are computationally prohibitive for
population synthesis, while purely secular codes lack the fidelity needed
for higher‑order dynamical systems (e.g.\ triples or cluster environments).
The modules documented here bridge these regimes by injecting carefully
book‑kept mass changes and dissipative forces into an otherwise
high‑precision $N$‑body integration.

The Roche‑lobe mass transfer operator:

\begin{itemize}[nosep,leftmargin=1.8em]
\item tracks \emph{all} linear momentum channels to machine precision;
\item offers four specific‑angular‑momentum prescriptions for systemic winds;
\item prevents numerical divergence near coalescence through tight merge guards;
\item adapts internal sub‑steps based on mass‑loss and dynamical
      time‑scale criteria; and
\item exposes an optional donor mass‑radius power law for adiabatic
      responses to mass loss.
\end{itemize}

%-------------------------------------------------------------------------------
\section{Physical Model and Mathematical Formulation}

\subsection{Roche‑Lobe Overflow (RLOF)}
\label{sec:RLOF}

\paragraph{Roche‑lobe radius.}
The Eggleton formula~\cite{Eggleton1983} for the donor’s
volume‑equivalent Roche radius,
\begin{equation}
R_{\mathrm L}
 = a\,
   \frac{0.49\,q^{2/3}}{0.60\,q^{2/3}+\ln(1+q^{1/3})},
\qquad
q\equiv\frac{M_{\mathrm d}}{M_{\mathrm a}},
\label{eq:Eggleton}
\end{equation}
is evaluated at run time to compute the overflow degree.

\paragraph{Mass‑loss law.}
The donor’s instantaneous mass‑loss rate follows
\citet{Ritter1988}:
\begin{equation}
\dot M_{\mathrm d}
 = -\dot M_0
   \exp\!\bigl[(R_{*}-R_{\mathrm L})/H_P\bigr],
\label{eq:Ritter}
\end{equation}
with the exponent clamped to $\le80$ to avoid floating‑point overflow
(\texttt{RLMT\_EXP\_CLAMP}).

\paragraph{Envelope veto.} If \texttt{rlmt\_skip\_in\_CE}=1\ (the default), the RLOF calculation is bypassed whenever the accretor resides inside the donor’s radius ($r<R_*$), allowing common‑envelope drag to dominate.

\paragraph{Non‑conservative transfer.}
A user‑set fraction $f_{\mathrm loss}\in[0,1]$ escapes as a wind,
\begin{equation}
\dot M_{\mathrm a}=-(1-f_{\mathrm loss})\,\dot M_{\mathrm d},\qquad
\dot M_{\mathrm wind}=f_{\mathrm loss}\,|\dot M_{\mathrm d}|.
\label{eq:noncon}
\end{equation}

\paragraph{Specific linear momentum of the wind.}
Four prescriptions can be selected at run time:
\begin{subequations}\label{eq:joptions}
\begin{align}
\mathbf v_{\mathrm loss} &= \mathbf v_{\mathrm d} & (\text{mode }0)\\
                         &= \mathbf v_{\mathrm a} & (\text{mode }1)\\
                         &= \mathbf v_{\mathrm cm}& (\text{mode }2)\\
                         &= \mathbf v_{\mathrm d} + f_j\,\mathbf e_\perp
                            \frac{j_{\rm orb}}{\lvert\mathbf r\rvert} & (\text{mode }3),
\end{align}
\end{subequations}
where $\mathbf e_\perp$ is any unit vector orthogonal to the separation
vector, $j_{\rm orb}=|\mathbf r\times\mathbf v|\,\mu^{-1}$ with reduced mass
$\mu=M_{\mathrm d}M_{\mathrm a}/(M_{\mathrm d}+M_{\mathrm a})$, and
$r=|\mathbf r|$ is the instantaneous separation.

\paragraph{Momentum bookkeeping and angular‑momentum correction.}
Accreted material arrives with the donor’s velocity, while wind
particles remove the momentum $m_{\mathrm wind}\mathbf v_{\mathrm loss}$.
After each RLOF step a minimal
velocity shift
\(
\delta\mathbf v=(\Delta\mathbf L\times\mathbf r_{\mathrm a})/(M_{\mathrm a}r_{\mathrm a}^2)
\)
is applied to the accretor to compensate for the angular momentum carried by
the transferred gas and keep the system’s total angular momentum unchanged.
The donor receives the opposite kick scaled by $M_{\mathrm a}/M_{\mathrm d}$ so
that linear momentum remains conserved.

\paragraph{Donor radius evolution (optional).}
If the donor carries attributes
\texttt{rlmt\_R\_slope}, \texttt{rlmt\_R\_ref\_mass},
\texttt{rlmt\_R\_ref\_radius}, its radius is updated via the power‑law
\[
R_{\mathrm d}(M)=R_{\mathrm ref}
\left(\frac{M}{M_{\mathrm ref}}\right)^{\alpha_R},
\]
enabling adiabatic or thermally driven mass‑radius responses.

%-------------------------------------------------------------------------------
\subsection{Common–Envelope (CE) Dynamical Friction}
\label{sec:ce_drag}

\paragraph{Drag acceleration.}
When the accretor resides inside the donor ($r<R_*)$ and a density profile is
available, a companion of mass $M_{\mathrm a}$ moving through envelope gas
experiences
\[
\mathbf a_{\rm DF}=
-\frac{4\pi\,G^2\,M_{\mathrm a}\,\rho}{v_{\rm rel}^3}\,
   I(\mathcal M)\,\mathbf v_{\rm rel},
\qquad
\mathcal M\equiv\frac{v_{\rm rel}}{c_s},
\]
where the dimensionless factor is
\begin{equation}
I(\mathcal M)=
\begin{cases}
\dfrac{1}{3}\mathcal M^3+\dfrac{1}{5}\mathcal M^5,&\mathcal M<0.02,\\[0.6em]
\dfrac{1}{2}\ln\!\dfrac{1+\mathcal M}{1-\mathcal M}-\mathcal M,
 & 0.02\le\mathcal M<1,\\[0.6em]
\ln\!\bigl(1/x_{\min}\bigr),&\mathcal M\ge1.
\end{cases}
\label{eq:I_prefactor}
\end{equation}
For $\mathcal M<1$ the implementation additionally caps the result at
$\ln(1/x_{\min})$,
ensuring $I(\mathcal M)\le\ln(1/x_{\min})$ in \emph{all} regimes.

\paragraph{Hydrodynamic accretion term (optional).}
If \texttt{ce\_Qd}$>$0,
a Bondi–Hoyle‐like contribution
\[
\mathbf a_{\rm GM}
=-\frac{\pi\,\rho\,R_{\mathrm acc}^2\,v_{\rm rel}\,Q_d}{M_{\mathrm a}}
\,\mathbf v_{\rm rel}
\]
is added.

\paragraph{Envelope structure.}
Density $\rho$ and sound speed $c_s$ are taken from either
(i) a power‑law $\rho=\rho_0s^{\alpha_\rho}$, $c_s=c_{s0}s^{\alpha_{c_s}}$
or
(ii) a user‑supplied tabulated profile $(s,\rho,c_s)$ loaded once via
\texttt{ce\_profile\_file}.  The table overrides the power‑law and is sampled
with log–log linear interpolation.  If neither option is provided the CE drag
step is skipped.

\paragraph{CFL‑like limiter.}
The velocity kick magnitude is limited to
$|\Delta\mathbf v|\le\texttt{ce\_kick\_cfl}\times c_s$ to preserve accuracy
in highly supersonic or stratified flows.

\paragraph{Numerical stability.}
The denominator $v_{\rm rel}^3$ is left \emph{unfloored}; instead the code
ensures $v_{\rm rel}$ never becomes arbitrarily small via the
sub‑step criterion $\Delta r/r\le\texttt{rlmt\_substep\_max\_dr}$ and likewise
limits mass changes through $\Delta M/M\le\texttt{rlmt\_substep\_max\_dm}$.
Users who require an explicit floor can modify the source accordingly.

%-------------------------------------------------------------------------------
\subsection{Magnetic Braking}
\label{sec:mb}

Low‑mass stars with convective envelopes lose spin angular momentum through
magnetised winds.  The implementation follows the Verbunt–Zwaan and Kawaler
torque law \citep{Verbunt1981,Kawaler1988}, applying a braking torque
antiparallel to the star's spin vector.

\paragraph{Torque law.} For a star of mass $M$, radius $R$, and angular
velocity $\omega = |\bm\Omega|$, the spin‑down torque is
\begin{equation}
\tau = -K R^{1/2} M^{-1/2} \omega^3,
\label{eq:mb_torque}
\end{equation}
where $K$ is a user‑set normalisation (parameter \texttt{mb\_K}, default
$2.7\times10^{47}$ in cgs).  The torque modifies the spin by
$\dot{\bm\Omega}=\tau\,\bm\Omega/(I\,\omega)$ with $I$ the particle's moment
of inertia.

\paragraph{Saturation.} If a particle specifies a saturation threshold
\texttt{mb\_omega\_sat} and $\omega>\omega_{\rm sat}$, the cubic dependence is
replaced with $\omega\,\omega_{\rm sat}^2$, yielding a constant torque at high
rotation rates.

\paragraph{Activation.} Magnetic braking is applied only when a particle sets
\texttt{mb\_on}=1 and has \texttt{mb\_convective}=1. Missing parameters,
non‑positive $I$, vanishing spin vectors, or non‑finite $M$ or $R$ all
bypass the update.

\begin{table}[h]
\centering\footnotesize
\caption{Magnetic braking parameters}
\label{tab:mb}
\begin{tabular}{@{}llll@{}}
\toprule
Name (scope) & Unit & Default & Purpose \\
\midrule
\texttt{mb\_K} (op) & cgs & $2.7\times10^{47}$ & Braking constant $K$\\
\texttt{mb\_on} (part) & bool & 0 & Enable magnetic braking\\
\texttt{mb\_convective} (part) & bool & 0 & Convective‑envelope flag\\
\texttt{mb\_omega\_sat} (part) & 1/t & $\infty$ & Saturation angular velocity\\
\texttt{I} (part) & mass\,length$^2$ & — & Moment of inertia\\
\texttt{Omega} (part) & 1/t & — & Spin angular frequency vector\\
\bottomrule
\end{tabular}
\end{table}

\subsection{Stellar Wind Mass Loss}
\label{sec:swml}

Isotropic winds remove mass from single stars according to the
Reimers prescription. For a star of mass $M$, luminosity $L$, and radius
$R$, the mass-loss rate is
\[
\dot M = -4\times10^{-13}\,\eta\,\frac{L}{L_\odot}\frac{R}{R_\odot}\frac{M_\odot}{M}
\;M_\odot\,\mathrm{yr}^{-1},
\]
where the dimensionless efficiency $\eta$ together with $L$ and $R$ are
specified per particle. Mass is removed isotropically with no linear-momentum
recoil; virtual particles are ignored and after mass loss the system is
recentred on the centre of mass. The prefactor, solar reference values, and
the year length can be adjusted via operator parameters
(Table~\ref{tab:swml}).

\begin{table}[h]
\centering\footnotesize
\caption{Stellar wind mass-loss parameters}
\label{tab:swml}
\begin{tabular}{@{}llll@{}}
\toprule
Name (scope) & Unit & Default & Purpose \\
\midrule
\texttt{swml\_eta} (part) & — & — & Wind efficiency $\eta$\\
\texttt{swml\_L}   (part) & luminosity & — & Stellar luminosity $L$\\
\texttt{swml\_R}   (part) & length & — & Stellar radius $R$\\[0.2em]
\texttt{swml\_const} (op) & $M_\odot$/yr & $4\times10^{-13}$ & Reimers prefactor\\
\texttt{swml\_Msun}  (op) & mass & 1 & Solar mass in code units\\
\texttt{swml\_Rsun}  (op) & length & 1 & Solar radius in code units\\
\texttt{swml\_Lsun}  (op) & luminosity & 1 & Solar luminosity in code units\\
\texttt{swml\_year}  (op) & time & 1 & Length of Julian year in code units\\
\bottomrule
\end{tabular}
\end{table}

%-------------------------------------------------------------------------------
\section{Algorithmic Flow}

Each operator call subdivides the requested time span into at least
\texttt{rlmt\_min\_substeps} internal steps.  The sub‑step size obeys the
limits $|\Delta M|/M\le\texttt{rlmt\_substep\_max\_dm}$ and
$|\Delta r|/r\le\texttt{rlmt\_substep\_max\_dr}$ to control mass transfer and
orbital motion.  An initial guard merges the pair if their separation falls
below $\varepsilon_{\rm merge}$, where
$\varepsilon_{\rm merge}=\texttt{merge\_eps}$ or
$0.5\min(R_{*},R_{\rm acc})$ when unspecified.

\begin{enumerate}[nosep]
\item \textbf{Pre‑check:} merge guard if $r\le\varepsilon_{\rm merge}$.
\item \textbf{Sub‑step loop} (adaptive):
  \begin{enumerate}[nosep]
    \item RLOF mass exchange and momentum bookkeeping.
    \item CE drag (if inside envelope).
    \item Secondary merge guard.
  \end{enumerate}
\item Purge zero‑mass particles.  If either the donor or accretor vanishes,
      the operator detaches automatically; finally the system is recentred on
      its centre of mass.
\end{enumerate}

%-------------------------------------------------------------------------------
\section{Run‑Time Parameters}
\label{sec:param_table}

\begin{table}[h]
\centering\footnotesize
\caption{Operator-level (\textit{op}) and particle-level (\textit{part}) parameters for the
Roche-lobe mass transfer effect.}
\label{tab:params}
\begin{tabular}{@{}lllll@{}}
\toprule
Name (scope) & Unit & Default & Purpose \\
\midrule
\texttt{rlmt\_donor}      (op)   & —        & — & Donor particle index\\
\texttt{rlmt\_accretor}   (op)   & —        & — & Accretor particle index\\
\texttt{rlmt\_Hp}         (part) & length   & — & Pressure‐scale height $H_P$\\
\texttt{rlmt\_mdot0}      (part) & M/t      & — & Reference rate $\dot M_0$\\[0.2em]
%
\texttt{rlmt\_loss\_fraction} (op) & —      & 0 & Wind mass fraction $f_{\mathrm loss}$\\
\texttt{jloss\_mode}      (op)   & int      & 0 & Choice in eqs.~\eqref{eq:joptions}\\
\texttt{jloss\_factor}    (op)   & —        & 1 & Scale factor $f_j$ (mode 3)\\
\texttt{rlmt\_skip\_in\_CE}(op)  & bool     & 1 & Disable RLOF if inside envelope\\[0.2em]
%
\texttt{rlmt\_substep\_max\_dm} (op) & —    & $10^{-3}$ & Max $|\Delta M|/M$ per sub‑step\\
\texttt{rlmt\_substep\_max\_dr} (op) & —    & $5\times10^{-3}$ & Max $|\Delta r|/r$ per sub‑step\\
\texttt{rlmt\_min\_substeps}    (op) & int  & 3 & Minimum sub‑steps per call\\[0.2em]
%
\texttt{rlmt\_R\_slope}      (part) & —     & 0 & Donor mass‑radius exponent $\alpha_R$\\
\texttt{rlmt\_R\_ref\_mass}  (part) & mass  & $M_{\rm init}$ & Reference mass $M_{\rm ref}$\\
\texttt{rlmt\_R\_ref\_radius}(part) & length& $R_{\rm init}$ & Reference radius $R_{\rm ref}$\\[0.2em]
%
\texttt{ce\_rho0}, \texttt{ce\_cs}  (op) & cgs & — & Power‑law normalisations\\
\texttt{ce\_alpha\_rho}, \texttt{ce\_alpha\_cs} (op) & — & 0 & Power‑law slopes\\
\texttt{ce\_profile\_file}  (op) & path & — & ASCII $(s,\rho,c_s)$ profile, overrides power law\\
\texttt{ce\_kick\_cfl}      (op) & — & 1 & Velocity‑kick limiter\\
\texttt{ce\_xmin}           (op) & — & $10^{-4}$ & Coulomb cutoff $x_{\min}$\\
\texttt{ce\_Qd}             (op) & — & 0 & Geometric drag coefficient\\[0.2em]
\texttt{merge\_eps}         (op) & length & $0.5\min(R_*,R_{\rm acc})$ & Merge radius\\
\bottomrule
\end{tabular}
\end{table}

%-------------------------------------------------------------------------------
\section{Post-Newtonian Relativistic Corrections}
Compact binaries also experience post-Newtonian forces that drive apsidal
precession and couple the eccentricity and inclination. We provide a dedicated
force \texttt{post\_newtonian} implementing the leading conservative $1$PN and
dissipative $2.5$PN terms for every particle pair.

For bodies with masses $m_i$ and $m_j$ separated by $\mathbf{r}$ and with
relative velocity $\mathbf{v}$, the force adds to the relative acceleration
\begin{align}
\mathbf{a}_{1\mathrm{PN}} &= -\frac{G m}{r^2 c^2}
   \Bigl[\mathbf{n}\bigl((1+3\eta)v^2 - 2(2+\eta)\frac{G m}{r}-\tfrac{3}{2}\eta\dot r^2\bigr)\Bigr]
   +\frac{G m}{r^2 c^2}(4-2\eta)\dot r\,\mathbf{v},\\
\mathbf{a}_{2.5\mathrm{PN}} &= \frac{8 G^2 m^2 \eta}{5 c^5 r^3}\Bigl[\dot r\Bigl(18 v^2 + \tfrac{2}{3}\frac{G m}{r}-25\dot r^2\Bigr)\mathbf{n}-\Bigl(6 v^2 - 2\frac{G m}{r}-15\dot r^2\Bigr)\mathbf{v}\Bigr],
\end{align}
where $m=m_i+m_j$, $\eta=m_i m_j/m^2$, $\mathbf{n}=\mathbf{r}/r$, and
$\dot r = \mathbf{v}\cdot\mathbf{n}$. These expressions correspond to
Eqs.~2.2b and~2.2f of \citet{Kidder1995}. The relative acceleration is
distributed to the two particles in proportion to their masses, preserving
linear momentum.

The only required effect parameter is the speed of light $c$, specified through
the field name \texttt{c}. Setting $c$ consistently with the simulation units
reproduces the classical perihelion advance and gravitational-wave driven
inspiral~\citep{Einstein1915, Peters1964, Kidder1995}.

\begin{table}[h]
\centering\footnotesize
\caption{Post-Newtonian effect parameter}
\label{tab:pn}
\begin{tabular}{@{}lll@{}}
\toprule
Field & Unit & Description \\
\midrule
\texttt{c} & length/t & Speed of light (required)\\
\bottomrule
\end{tabular}
\end{table}

%-------------------------------------------------------------------------------
\section{Discussion and Future Extensions}
\label{sec:future}
Future work may add
(i) thermally driven winds and
(ii) direct coupling to tabulated stellar‑evolution tracks
so that $R_*(M)$ and $H_P(M)$ evolve self‑consistently.

%-------------------------------------------------------------------------------
\section{Conclusions}
We have brought the LaTeX documentation in line with the
\texttt{C} implementations of the
\texttt{roche\_lobe\_mass\_transfer}, \texttt{magnetic\_braking},
\texttt{stellar\_wind\_mass\_loss}, and \texttt{post\_newtonian} modules.
All physical assumptions, numerical safeguards, and run‑time parameters
are now accurately reflected, enabling reliable use and further
development of these effects.

%-------------------------------------------------------------------------------
\bibliographystyle{plainnat}
\begin{thebibliography}{}
\bibitem[Eggleton(1983)]{Eggleton1983}
  Eggleton,~P.\ 1983, \emph{ApJ}, 268, 368.
\bibitem[Kawaler(1988)]{Kawaler1988}
  Kawaler,~S.~D.\ 1988, \emph{ApJ}, 333, 236.
\bibitem[Ostriker(1999)]{Ostriker1999}
  Ostriker,~E.\ 1999, \emph{ApJ}, 513, 252.
\bibitem[Peters(1964)]{Peters1964}
  Peters,~P.\ 1964, \emph{Phys.\ Rev.}, 136, B1224.
\bibitem[Kidder(1995)]{Kidder1995}
  Kidder,~L.~E.\ 1995, \emph{Phys.\ Rev.\ D}, 52, 821.
\bibitem[Einstein(1915)]{Einstein1915}
  Einstein,~A.\ 1915, \emph{Preuss.\ Akad.\ Wiss.\ Berlin}, 831.
\bibitem[Ritter(1988)]{Ritter1988}
  Ritter,~H.\ 1988, \emph{A\&A}, 202, 93.
\bibitem[Verbunt \& Zwaan(1981)]{Verbunt1981}
  Verbunt,~F. \& Zwaan,~C.\ 1981, \emph{A\&A}, 100, L7.
\end{thebibliography}

\end{document}
