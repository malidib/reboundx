\section{Stellar Wind Mass Loss}
\label{sec:stellar_wind}

Low- and intermediate-mass stars lose mass through steady winds during
post--main-sequence evolution.  A commonly used empirical
prescription is the Reimers law\cite{Reimers1975}
\begin{equation}
  \dot M = -4\times10^{-13}\,\eta\,\frac{L}{L_\odot}\frac{R}{R_\odot}\frac{M_\odot}{M}\quad M_\odot\,\mathrm{yr}^{-1},
\end{equation}
where $L$ and $R$ are the stellar luminosity and radius and $\eta$ is a
free efficiency parameter.  This operator applies mass loss according
to this equation each timestep.  The user specifies $L$, $R$ and
$\eta$ as particle parameters.  The prefactor and the solar-unit
conversions can be adjusted through operator parameters.

The removed mass is simply subtracted from the particle's mass.
Whenever the remaining mass would become negative, the particle's mass
is set to zero and mass loss stops.  After updating the masses, the
simulation is moved back to the centre of mass.

\bibliographystyle{plainnat}
\begin{thebibliography}{99}
\bibitem[Reimers(1975)]{Reimers1975} Reimers, D. 1975, Mem. Soc. R. Sci. Li\`ege, 8, 369
\end{thebibliography}
